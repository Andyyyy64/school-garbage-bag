\documentclass[platex,a4paper,12pt,dvipdfmx,aspectratio=169]{beamer}
\usetheme{default}
\usepackage{xcolor}

\title{beamer プレゼンテーションサンプル}
\subtitle{LI01 Computer Literacy}
\author{Computer Literacy 教員}

\institute{会津大学}
\begin{document}

\frame{
\titlepage
}

\frame{\frametitle{Title 1}
このページの本文
}

\begin{frame}{Title 2}
frame環境で作成したページ
\end{frame}

\frame{\frametitle{Itemize}
\begin{itemize}
\item 箇条書き
\item itemize や enumerateを使う
\begin{enumerate}
\item 入れ子も可能
\item 字下げ (インデント)される
\end{enumerate}
\end{itemize}
}

\frame{\frametitle{tex $\rightarrow $ dvi $\rightarrow $ pdf}
\begin{enumerate}
\item  (tex $\rightarrow $ dvi): platex slide.tex 
\item  (dvi $\rightarrow $ pdf ): dvipdfmx slide.dvi
\end{enumerate} 
}

\begin{frame}{Equation}
\begin{itemize}
\item  式もいれることができる
\begin{equation} \pi = 4\arctan(1) \end{equation}
\item \textcolor{violet}{文字色も変えることができる}
\item \colorbox{lightgray}{背景色も変えることができる}
\end{itemize}
\end{frame}

\begin{frame}{Figure and Table}

\includegraphics[width=4zw]{/home/course/literacy/pub/sample/FIGURE/uoa_logo.png}

\begin{itemize}
\item  図も入れることができる
\item  表も入れることができる
\end{itemize}

\begin{tabular}{rclp{5em}}
  \hline
  Column1    & Column2 & Column3 & Column4\\
  \hline
   Right    & Left & Center &  MMMMM\\
  Dd     & E & F & G\\
  \hline
\end{tabular}


\end{frame}


\end{document}
