\documentclass[platex,a4paper,12pt,dvipdfmx]{jsarticle}
\title{{\LaTeX}演習課題}
\author{s1300107ロッシアンディ拓也}
\begin{document}
\maketitle
この文書は{\LaTeX}の演習課題として作成されたものである。

\section{\LaTeX}
LATEX はコンピュータ理工学分野で広く利用されている学術論文や技術文書を作成する
ための文書整形システム(組版処理システム)である。Stanford University の Donald E.
Knuth が 1978 年に発表した組版エンジン・組版言語の TeX を核として、DEC (Digital
Equipment Corporation) のコンピュータサイエンティスト Leslie Lamport によって機
能強化する形で開発された。
コンピュータリテラシーのハンドアウト [1] によれば、LATEX は以下のような要素を簡
単に取り扱うことができるとされている。
\begin{itemize}
\item 数式
\item 箇条書き
\item 表
\item 画像の貼り付け
\item 引用文献
\item プレゼンテーション資料作成
\end{itemize}
\section{落体の運動}
\subsection{万有引力}
落体がなぜ落ちるかといえば、それは重力が作用しているからに他ならない。質量 $M$
の物体 $A$ と質量 $m$ の物体 $B$ の間に作用する万有引力の大きさは
\[F = G\frac{Mm}{r^2} \]
であり、その向きは 2 物体の中心を結ぶベクトル$\vec{r} = \vec{r_A} - \vec{e_B}$の向きである。ただし、$G$
は重力定数,$r = |\vec {r}|$は 2 物体間の距離である。
物体に力が作用すると、物体にはその力に比例した加速度$\alpha$ が生じる。その比例係数は質量の逆数であり、\begin{equation}\alpha = \frac{F}{m} = G\frac{M}{r^2} \end{equation} となる。従って、加速度は$M$に寄らないことがわかる。
\subsection{地球上での落体運動}
地球上での落体運動を考えてみよう。地球の質量を $M$ 、半径を $R$ とし、地表からの高
さ $h$ にある物体(例えばりんご)の質量を $m$ とする。
地球の中心から物体までの距離は$r = h + R$が成り立つので、物体に対する加速度の
大きさは\begin{equation}\alpha = G\frac{M}{(h+R)^2}\end{equation}
である。$R$に対して$h$が十分小さいことを用いると、式 (2) の右辺は
\begin{eqnarray}G\frac{M}{(h+R)^2} &=& G\frac{M}{R^2}\times(1+\frac{h}{R})^{-2}\\ 
                                   &=& G\frac{M}{R^2}(1-2\frac{h}{R}+...)\\
                                &\sim&G\frac{M}{R^2}
                                \end{eqnarray}と近似できる。これを重力加速度 $g$ と呼び、重力定数 $G$、地球質量 $M$ と半径 $R$ から約
9.8 $m/s^2$ という値を得る。地表面付近で質量$m$の物体に働く重力は$mg$となる。
\begin{thebibliography}{99}
\bibitem{lit} 2022年度コンピュータリテラシー, コンピュータリテラシー教員グループ, 2022.
\end{thebibliography}
\end{document}


